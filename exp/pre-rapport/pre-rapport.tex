\documentclass[a4paper,12pt, twoside]{article}
\usepackage[utf8x]{inputenc} %commentaire
\usepackage[francais]{babel}
\usepackage[T1]{fontenc}
\usepackage{graphicx}
\usepackage{wrapfig}
\usepackage{algpseudocode}

% Style Page
\pagestyle{headings}

% Title Page
\title{
  \textbf{Mini-Rapport}
  \\[1cm]
  Exploration de la notion de méta-apprentissage
  \\[1.3cm]
  \textit{
  Dans quelle mesure un système apprenant peut prendre conscience de ses performances
  et altérer son comportement ?}
}


\author{
  \\
  \\
  \\
  \\
  \\
  \\
  \\
  \\
  \\
  Matthieu Zimmer \\
  avec \\
  Yann Boniface, Alain Dutech, Nicolas Rougier }
\date{premier semestre 2012}


\begin{document}
\maketitle


%\begin{abstract}
%\end{abstract}


%\part{Partie}

\newpage
\section{Test}


%Le savoir acquis dans un réseau connexionniste reste toujours 
%de la connaissance dans le réseau plutôt que des connaissances 
%pour le réseau. 
%\newline
%Clark and Karmiloff-Smith's [Clark, A., \& Karmiloff-Smith, A. (1993)]

%Lorsqu'on est conscient de quelque chose, on est aussi conscient d'être conscient.
%\newline
%Higher-Order Thought Theory [Rosenthal, D. (1997).]

%Ces réseaux peuvent devenir extrêmement sensible à des régularités contenues dans 
%leur environnement d'entrée-sortie, mais ils n'exposent jamais la capacité à 
%accéder et manipuler cette connaissance que la connaissance. Ce savoir ne peut
%être qu'exprimé à travers l'exécution de la tâche à laquelle le réseau à était entrainé.
%\newline
%Consciousness and metarepresentation : A computational sketch
%[ Alex Cleeremans, Bert Timmermans, Antoine Pasquali . (2007)]



\end{document}          
