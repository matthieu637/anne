\documentclass[11pt]{beamer}

\usepackage[T1]{fontenc}
\usepackage[utf8x]{inputenc}
\usepackage[frenchb]{babel}
\usepackage{amsmath}
\usepackage{lmodern}
\usepackage{xcolor}
\usepackage{graphicx}
\usepackage{pstricks}

% \usepackage{lmodern}

% \usepackage[noframe]{showframe}


\usetheme{Singapore}
% \useoutertheme{smoothbars}
% \useinnertheme[shadow=true]{rounded}
% \usecolortheme{orchid}
% \usecolortheme{whale}


\setbeamertemplate{navigation symbols}{}

\definecolor{cloneBlue}{rgb}{0.2,0.2,0.698}


\newenvironment{changemargin}[2]{%
  \begin{list}{}{%
    \setlength{\topsep}{0pt}%
    \setlength{\leftmargin}{#1}%
    \setlength{\rightmargin}{#2}%
    \setlength{\listparindent}{\parindent}%
    \setlength{\itemindent}{\parindent}%
    \setlength{\parsep}{\parskip}%
  }%
  \item[]}{\end{list}} 

\newenvironment{noitemize}
{\begin{list}{}{%
\setlength{\labelwidth}{0em}% largeur de la boite englobant l'étiquette
\setlength{\labelsep}{2pt}% espace entre l'entrée de l'item et l'étiquette
\setlength{\leftmargin}{0pt}% marge de gauche
\renewcommand{\makelabel}{\small\color{cloneBlue}{\textbullet}}}}%
{\end{list}}

\newenvironment{minusitemize}
{\begin{list}{}{%
\setlength{\labelwidth}{0em}% largeur de la boite englobant l'étiquette
\setlength{\labelsep}{2pt}% espace entre l'entrée de l'item et l'étiquette
\setlength{\leftmargin}{-15pt}% marge de gauche
\renewcommand{\makelabel}{\small\color{cloneBlue}{\textbullet}}}}%
{\end{list}}


\setbeamersize{text margin left=0.75cm}
\setbeamersize{text margin right=0.75cm}


\author{Yann Boniface, Alain Dutech, Nicolas Rougier, Matthieu Zimmer}
\title{Exploration de la notion de méta-apprentissage}
\subtitle[\ldots]{Dans quelle mesure un système apprenant peut « prendre conscience » de ses performances et altérer son comportement ?}
\institute{Loria}
\date{\today}
\logo{\includegraphics[height=6mm]{logo.png}}

\begin{document}

\maketitle



\begin{frame}
 \frametitle{Inspiration : Conscience et méta-représentations}
 \framesubtitle{Articles}
 
 \begin{itemize}
  \item Consciousness and metarepresentation : A computational sketch
  \newline [ Alex Cleeremans, Bert Timmermans, Antoine Pasquali ]
  \item Know thyself : Metacognitive networks and mesures of consciousness
  \newline [ Antoine Pasquali, Bert Timmermans, Alex Cleeremans ]
 \end{itemize}
\end{frame}



\begin{frame}
  \begin{center}{\Large Plan }\end{center}
  \tableofcontents[hideallsubsections]
\end{frame}


\section{Dupliquer le premier réseau}

\begin{frame}
  \frametitle{La base de départ / Rappel}
  \begin{center}
    \begin{pspicture}(0,5)
      \rput[B](0,0){\includegraphics[height=150px]{../cleeremans_2007/digit_reco/digit_reco.png}}
      \rput[B](3,0.5){\tiny{Consciousness and metarepresentation : A computational sketch}}
    \end{pspicture}
  \end{center}

  \begin{center}
    \begin{tabular}{lr}
    \begin{minipage}{150px}
      
      \footnotesize\begin{minusitemize}
      \item 20 entrées (représentant les chiffres)
      \item le premier réseau discrime les 10 chiffres
      \item winner-take-all sur les sorties
      \end{minusitemize}

      \end{minipage}
      &
      \begin{minipage}{170px}
      \footnotesize\begin{noitemize}
      \item la couche cachée du premier réseau sert d'entrée au second
      \item le second réseau apprend à dupliquer toutes les couches du premier
      \end{noitemize}
      
    \end{minipage}
    \end{tabular}
  \end{center}
  
\end{frame}

\begin{frame}
  \frametitle{Résultat sur la base d'entrée de l'article}
  \begin{center}
  \begin{tabular}{cc}
  \hspace*{-1cm}
   \includegraphics[width=185px]{../cleeremans_2007/digit_reco/rms.png}
   &
   \hspace*{-0.5cm}
   \includegraphics[width=170px]{../cleeremans_2007/digit_reco/rms_ffa.png}
  \end{tabular}
  \end{center}

\begin{itemize}
 \item la couche cachée et la couche de sortie ne posent aucun problèmes d'apprentissage
 \item les performances du second réseau dépendent principalement de sa capacité à reproduire les entrées
\end{itemize}
\end{frame}


\begin{frame}
  \frametitle{Dans la couche cachée du FoN : les entrées du SoN}
  \begin{center}
   \includegraphics[width=185px]{../cleeremans_2007/digit_reco/discretize_cloud.png}
  \end{center}

\begin{itemize}
 \item stabilisation très rapidement (autour de la 50\up{ième} époque en moyenne)
 \item entrées peu variables et stable favorisant son apprentissage
\end{itemize}
\end{frame}



\begin{frame}
  \frametitle{Passage à niveau}
  \begin{center}
  \includegraphics[height=150px]{../cleeremans_2007/digit_reco/schema_handwritten.png}
  \end{center}

  \begin{center}
  \begin{tabular}{lr}
  \begin{minipage}{150px}
    
    \footnotesize\begin{minusitemize}
     \item chiffres manuscrits sur 256 neurones d'entrées
     \item Le premier réseau discrime \textbf{1600 chiffres}
     \item Winner-take-all sur les sorties
    \end{minusitemize}

    \end{minipage}
    &
    \begin{minipage}{170px}
    \footnotesize\begin{noitemize}
     \item la couche cachée du premier réseau sert d'entrée au second
     \item le second réseau apprend à dupliquer toutes les couches du premier
    \end{noitemize}
    
    \end{minipage}
  \end{tabular}

  \end{center}
  
\end{frame}


\begin{frame}
  \frametitle{Résultats}
  \begin{center}
  \begin{tabular}{cc}
  \hspace*{-1cm}
   \includegraphics[width=175px]{../cleeremans_2007/digit_reco/rms_ffa.png}
   &
   \hspace*{-0.5cm}
   \includegraphics[width=170px]{../cleeremans_2007/digit_reco/rms_handwritten_ffa.png}
  \end{tabular}
  \end{center}

\begin{itemize}
 \item la couche cachée et la couche de sortie s'apprennent toujours bien
 \item il n'arrive plus à dupliquer les entrées
 \item ça fonctionne dans l'article car il n'y a que 10 entrées différentes
\end{itemize}
\end{frame}


\begin{frame}
  \frametitle{Dans la couche cachée du FoN : les entrées du SoN}
  \begin{center}
   \includegraphics[height=160px]{../cleeremans_2007/digit_reco/discretize_minhand.png}
  \end{center}

\begin{itemize}
 \item il existe différentes valeurs de la couche cachée, représentant le même nombre
 \item une même valeur discrétisée peut correspondre à  plusieurs couleurs
\end{itemize}
\end{frame}


\begin{frame}
  \frametitle{Représentation interne}
  
    \begin{center}
  \begin{tabular}{cc}
  \hspace*{-1cm}
   \includegraphics[width=130px]{../cleeremans_2007/digit_reco/schema_handwritten.png}
   &
   \includegraphics[width=150px]{../pre-rapport/metarepre.png}
  \end{tabular}
  \end{center}

\begin{itemize}
 \item il n'est pas si mauvais
 \item factorisation et perte d'information dans la couche cachée du FoN
\end{itemize}
\end{frame}



\begin{frame}
  \frametitle{Changement de tâche avec blocage de l'apprentissage}
  
    \begin{center}
  \begin{tabular}{cc}
  \hspace*{-1cm}
   \includegraphics[width=130px]{../cleeremans_2007/digit_reco/schema_handwritten.png}
   &
   \includegraphics[width=190px]{../cleeremans_2007/digit_reco/err_handwritten_relearn_2.png}
  \end{tabular}
  \end{center}
\end{frame}


% ----------------------------------------------------------------------------------------------------
% ----------------------------------------------------------------------------------------------------
% ----------------------------------------------------------------------------------------------------


\begin{frame}
\tableofcontents[hideallsubsections]
\end{frame}


\section{Amélioration de l'apprentissage}

\begin{frame}
  \frametitle{La base de départ / Rappel}
  \begin{center}
    \begin{pspicture}(0,5)
      \rput[B](0,0){\includegraphics[height=150px]{../cleeremans_2007/digital_reco/schema.png}}
      \rput[B](3,0.5){\tiny{Consciousness and metarepresentation : A computational sketch}}
    \end{pspicture}
  \end{center}

  \begin{center}
    \begin{tabular}{lr}
    \begin{minipage}{150px}
      
      \footnotesize\begin{minusitemize}
      \item 7 entrées (afficheur digital)
      \item le premier réseau discrime les 10 chiffres
      \item winner-take-all sur les sorties
      \end{minusitemize}

      \end{minipage}
      &
      \begin{minipage}{170px}
      \footnotesize\begin{noitemize}
      \item la couche cachée du premier réseau sert d'entrée au second
      \item le second réseau apprend à parier sur la qualité de la réponse du premier
      \end{noitemize}
      
    \end{minipage}
    \end{tabular}
  \end{center}
  
\end{frame}



\begin{frame}
  \frametitle{Résultat sur la base d'entrée de l'article}
  \begin{center}
  Performances de classification
  \begin{tabular}{cc}
  \hspace*{-1cm}
   \includegraphics[width=175px]{../cleeremans_2007/digital_reco/perf_wag.png}
   &
   \hspace*{-0.3cm}
   \includegraphics[width=175px]{../cleeremans_2007/digital_reco/perf_boost.png}
  \end{tabular}

  \begin{tabular}{lr}
  \begin{minipage}{170px}
    \hspace*{-0.2cm}
    \footnotesize Paramètre du second réseau :
    \footnotesize
      \begin{minusitemize}
      \item poids initilisés sur [-0.25; 0.25]
      \item momentum : 0
      \item non exploitable
      \end{minusitemize}
    \end{minipage}
    &
    \begin{minipage}{170px}
    \hspace*{0.2cm}
    \footnotesize Paramètre du second réseau :
    \footnotesize\begin{noitemize}
     \item poids initilisés sur [-1; 1]
     \item momentum : 0.5
     \item exploitable
    \end{noitemize}
    
    \end{minipage}
  \end{tabular}
  \end{center}
\end{frame}




\begin{frame}
  \frametitle{Passage à niveau}
  \begin{center}
  \includegraphics[height=150px]{base.png}
  \end{center}

  \begin{center}
  \begin{tabular}{lr}
  \begin{minipage}{150px}
    
    \footnotesize\begin{minusitemize}
     \item chiffres manuscrits sur 256 neurones d'entrées
     \item Le premier réseau discrime 1600 chiffres
     \item Winner-take-all sur les sorties
    \end{minusitemize}

    \end{minipage}
    &
    \begin{minipage}{170px}
    \footnotesize\begin{noitemize}
     \item la couche cachée du premier réseau sert d'entrée au second
     \item le second réseau apprend à parier sur la qualité de la réponse du premier
    \end{noitemize}
    
    \end{minipage}
  \end{tabular}

  \end{center}
  
\end{frame}



\begin{frame}
  \frametitle{Architectures de feedback}

  \hspace*{-0.8cm}
  \begin{tabular}{c|c|c}
  \begin{minipage}[b]{.33\linewidth}
   \includegraphics[width=100px]{base.png}
   \\
   \tiny Second neurone le plus élevé quand pari bas
   \end{minipage}
   
   &
   \begin{minipage}[b]{.33\linewidth}
      \includegraphics[width=100px]{../pre-presentation/nth_wta.png}
      \\
   \tiny Second réseau enregistre l'indice (par activation) du neurone contenant la bonne réponse
    \end{minipage}
   &
   \begin{minipage}[b]{.33\linewidth}
         \includegraphics[width=100px]{base.png}
      \\
   \tiny Second réseau contrôle le taux d'apprentissage et le momentum du premier réseau
    \end{minipage}
    \\
    \hline
       \begin{minipage}[b]{.33\linewidth}
         \includegraphics[width=100px]{../pre-presentation/merging.png}
      \\
   \tiny Les sorties du second réseaux deviennent des entrées supplémentaires au premier
    \end{minipage}
    &
    \begin{minipage}[b]{.33\linewidth}
         \includegraphics[width=100px]{../pre-presentation/thrid.png}
      \\
    \tiny 3ème réseau de perceptron
    \end{minipage}
    &
        \begin{minipage}[b]{.33\linewidth}
         \includegraphics[width=100px]{../pre-presentation/thrid_hidden.png}
      \\
    \tiny 3ème réseau de perceptron sur la couche cachée
    \end{minipage}

  \end{tabular}


\end{frame}


\begin{frame}
  \frametitle{Résultat concluants}
  \framesubtitle{3ème réseau de perceptron}
  
  \begin{tabular}{cc}
  \hspace*{-1cm}
   
   \includegraphics[width=175px]{../pre-rapport/third_net.png}
   &
   \hspace*{-0.3cm}
   \includegraphics[width=175px]{../pre-rapport/third_net_hidden.png}
  \end{tabular}  
\end{frame}


\begin{frame}
  \frametitle{Limitations}
  Le nombre de neurone dans la couche cachée du FoN déterminent beaucoup les performances :
  \begin{itemize}
   \item si il est ``normal'', le second réseau n'arrivera pas à parier efficacement, et se contente de 
   parier haut
   \item si il est élevé, le second réseau parie correctement, mais le premier réseau est détérioré.
   \newline
   De sorte que même avec l'augmentation de performance du SoN, il ne dépasse pas un FoN bien réglé
  \end{itemize}

\end{frame}

\begin{frame}
  \frametitle{Limitations}
  \begin{center}
    \includegraphics[height=220px]{../cleeremans_2007/digit_reco/bad_full.png}
  \end{center}
\end{frame}

% \begin{frame}
%   \frametitle{Conclusion}
%  En performances
%  
%  En terme d'interprétation
% \end{frame}




% \section{Transfert de tâche}
% \begin{frame}
% 
% \end{frame}
% 
% \begin{frame}
% 
% \end{frame}
% \section{Conscience}
% \begin{frame}
% 
% \end{frame}




\end{document}


