\documentclass[11pt]{beamer}

\usepackage[T1]{fontenc}
\usepackage[utf8x]{inputenc}
\usepackage[frenchb]{babel}
\usepackage{amsmath}
\usepackage{lmodern}
\usepackage{xcolor}
\usepackage{graphicx}
% \usepackage{lmodern}

% \usepackage[noframe]{showframe}


\usetheme{Singapore}
% \useoutertheme{smoothbars}
% \useinnertheme[shadow=true]{rounded}
% \usecolortheme{orchid}
% \usecolortheme{whale}


\setbeamertemplate{navigation symbols}{}

\definecolor{cloneBlue}{rgb}{0.2,0.2,0.698}


\newenvironment{changemargin}[2]{%
  \begin{list}{}{%
    \setlength{\topsep}{0pt}%
    \setlength{\leftmargin}{#1}%
    \setlength{\rightmargin}{#2}%
    \setlength{\listparindent}{\parindent}%
    \setlength{\itemindent}{\parindent}%
    \setlength{\parsep}{\parskip}%
  }%
  \item[]}{\end{list}} 

\newenvironment{noitemize}
{\begin{list}{}{%
\setlength{\labelwidth}{0em}% largeur de la boite englobant l'étiquette
\setlength{\labelsep}{2pt}% espace entre l'entrée de l'item et l'étiquette
\setlength{\leftmargin}{0pt}% marge de gauche
\renewcommand{\makelabel}{\small\color{cloneBlue}{\textbullet}}}}%
{\end{list}}

\newenvironment{minusitemize}
{\begin{list}{}{%
\setlength{\labelwidth}{0em}% largeur de la boite englobant l'étiquette
\setlength{\labelsep}{2pt}% espace entre l'entrée de l'item et l'étiquette
\setlength{\leftmargin}{-20pt}% marge de gauche
\renewcommand{\makelabel}{\small\color{cloneBlue}{\textbullet}}}}%
{\end{list}}


\setbeamersize{text margin left=0.75cm}
\setbeamersize{text margin right=0.75cm}


\author{Yann Boniface, Alain Dutech, Nicolas Rougier, Matthieu Zimmer}
\title{Exploration de la notion de méta-apprentissage}
\subtitle[\ldots]{Dans quelle mesure un système apprenant peut « prendre conscience » de ses performances et altérer son comportement ?}
\institute{Loria}
\date{\today}
\logo{\includegraphics[height=6mm]{logo.png}}

\begin{document}

\maketitle



\begin{frame}\transwipe
 \frametitle{Inspiration : Conscience et méta-représentations}
 \framesubtitle{Articles}
 
 \begin{itemize}
  \item Consciousness and metarepresentation : A computational sketch
  \newline [ Alex Cleeremans, Bert Timmermans, Antoine Pasquali ]
  \item Know thyself : Metacognitive networks and mesures of consciousness
  \newline [ Antoine Pasquali, Bert Timmermans, Alex Cleeremans ]
 \end{itemize}
\end{frame}



\begin{frame}\transwipe
  \begin{center}{\Large Plan }\end{center}
  \tableofcontents[hideallsubsections]
\end{frame}

\section{Amélioration de l'apprentissage}

\begin{frame}\transwipe
  \frametitle{La base de départ / Rappel}
  \begin{center}
  \includegraphics[height=150px]{../cleeremans_2007/digital_reco/schema.png}
  \end{center}

  \begin{center}
  \begin{tabular}{lr}
  \begin{minipage}{150px}
    
    \footnotesize\begin{minusitemize}
     \item 7 entrées (afficheur digital)
     \item le premier réseau discrime les 10 chiffres
     \item winner-take-all sur les sorties
    \end{minusitemize}

    \end{minipage}
    &
    \begin{minipage}{170px}
    \footnotesize\begin{noitemize}
     \item la couche cachée du premier réseau sert d'entrée au second
     \item le second réseau apprend à parier sur la qualité de la réponse du premier
    \end{noitemize}
    
    \end{minipage}
  \end{tabular}

  \end{center}
  
\end{frame}



\begin{frame}\transwipe
  \frametitle{Résultat sur la base d'entrée de l'article}
  \begin{center}
  \begin{tabular}{cc}
  \hspace*{-1cm}
   \includegraphics[width=175px]{../cleeremans_2007/digital_reco/perf_wag.png}
   &
   \hspace*{-0.3cm}
   \includegraphics[width=175px]{../cleeremans_2007/digital_reco/perf_boost.png}
  \end{tabular}

  \begin{tabular}{lr}
  \begin{minipage}{170px}
    e
    \footnotesize
      \begin{minusitemize}
      \item 7 entrées (afficheur digital)
      \item le premier réseau discrime les 10 chiffres
      \item non exploitable
      \end{minusitemize}
    \end{minipage}
    &
    \begin{minipage}{170px}
%     \includegraphics[width=180px]{../cleeremans_2007/digital_reco/perf_boost.png}
    \footnotesize\begin{noitemize}
     \item la couche cachée du premier réseau sert d'entrée au second
    \end{noitemize}
    
    \end{minipage}
  \end{tabular}

  \end{center}
  
\end{frame}




\begin{frame}\transwipe
  \frametitle{Passage à niveau}
  \begin{center}
  \includegraphics[height=150px]{base.png}
  \end{center}

  \begin{center}
  \begin{tabular}{lr}
  \begin{minipage}{150px}
    
    \footnotesize\begin{minusitemize}
     \item chiffres manuscrits sur 256 neurones d'entrées
     \item Le premier réseau discrime 1600 chiffres
     \item Winner-take-all sur les sorties
    \end{minusitemize}

    \end{minipage}
    &
    \begin{minipage}{170px}
    \footnotesize\begin{noitemize}
     \item la couche cachée du premier réseau sert d'entrée au second
     \item le second réseau apprend à parier sur la qualité de la réponse du premier
    \end{noitemize}
    
    \end{minipage}
  \end{tabular}

  \end{center}
  
\end{frame}

\begin{frame}\transwipe
  \frametitle{Limitation}

\end{frame}

\begin{frame}\transwipe
  \frametitle{Conclusion}
 En performances
 
 En terme d'interprétation
\end{frame}


\section{Transfert de tâche}
\begin{frame}\transwipe

\end{frame}

\begin{frame}\transwipe

\end{frame}
\section{Conscience}
\begin{frame}\transwipe

\end{frame}

% \section[Sujet 1 : Jeu vidéo]{Sujet 1 : Jeu vidéo}
% 
% % \subsection[titre cour2t1]{titre long long long 1}
% 
% 
% \subsection[ ]{ }
% \begin{frame}\transwipe
%  \frametitle{Sujet 1 : Jeu vidéo C++ en pseudo-3D (isométrique)}
%  \framesubtitle{Introduction}
%   \begin{itemize}[<+->]
%     \item Difficulté : de 3 à 4 étoiles
%       \begin{itemize}[<+->]
% \item C++ ( pointeurs, références, mémoire, templates, règles, ...)
% \item Bibliotèques : Boost / CEGUI / SFML
% \item Algorithmes : collision, pathfinding, affichage, IA, déplacement, ...
% \item Portalibité
% \item Durée \& Graphisme
% \item Echanges client/serveur
%       \end{itemize}
%     \item Durée : de 3 à 5 étoiles
%     \item Langages
%       \begin{itemize}[<+->]
% \item C++
% \item OpenGL
% \item Lua
% \item XML
% \item PostgreSQL
%       \end{itemize}
%   \end{itemize}
% \end{frame}
% 
% 
% \begin{frame}\transwipe
%  \frametitle{Sujet 1 : Jeu vidéo C++ en pseudo-3D (isométrique)}
%  \framesubtitle{Jeu 2D}
%  \begin{center}
%   \includegraphics[height=200px]{data/2D_1.jpg}
%  \end{center}
% \end{frame}
% 
% \begin{frame}\transwipe
%  \frametitle{Sujet 1 : Jeu vidéo C++ en pseudo-3D (isométrique)}
%  \framesubtitle{Jeu 3D}
%  \begin{center}
%   \includegraphics[height=200px]{data/3D_1.jpg}
%  \end{center}
% \end{frame}
% 
% 
% \begin{frame}\transwipe
%  \frametitle{Sujet 1 : Jeu vidéo C++ en pseudo-3D (isométrique)}
%  \framesubtitle{Perspective Isométrique}
%  \includegraphics[height=196px]{data/wakfu_screen3.jpg}
% \end{frame}
% 
% \begin{frame}\transwipe
%  \frametitle{Sujet 1 : Jeu vidéo C++ en pseudo-3D (isométrique)}
%  \framesubtitle{SFML vs CEGUI}
%  \includegraphics[height=196px]{data/wakfu_screen3.jpg}
% \end{frame}
% 
% \begin{frame}\transwipe
%  \frametitle{Sujet 1 : Jeu vidéo C++ en pseudo-3D (isométrique)}
%  \framesubtitle{Découpage}
%  \begin{center}
%  \includegraphics[height=196px]{data/S1_decoup.png}
%   \end{center}
% \end{frame}
% 
% \begin{frame}\transwipe
%  \frametitle{Sujet 1 : Jeu vidéo C++ en pseudo-3D (isométrique)}
%  \framesubtitle{Intérêts}
%   \begin{itemize}[<+->]
%     \item Prise en main d'un moteur graphique avancé (SFML)
%     \item Prise en main d'une GUI avancée (CEGUI)
%     \item Compréhension des mécanismes d'échanges entre GUI-moteur graphique
%     \item Découverte d'architectures pour jeux-vidéos
%     \item Mise en place de protocoles de communication personnalisés ( si serveur )
%     \item Equipe pluridisciplinaire (graphistes, scénaristes, ingé. son, ...)
%     \item Beaucoup d'autres choses dépendant du type de jeu choisi
%   \end{itemize}
% \end{frame}
% 
% \begin{frame}\transwipe
%  \frametitle{Sujet 1 : Jeu vidéo C++ en pseudo-3D (isométrique)}
%  \framesubtitle{Conclusion}
%   \begin{itemize}[<+->]
%     \item Très enrichissant
%     \item Possibilitées de financement ( publicité, F2P, services, ... )
%     \item Long et dépendant d'autrui
%     \item \bf{Originalité}
%   \end{itemize}
%   
%    \uncover<5->{
%     \begin{block}{ Chances de réussite }
% 60\%
%     \end{block}
%    }
%    
% \end{frame}
% 
% \section[Sujet 2 : IHM]{Sujet 2 : IHM}
% 
% \subsection[ ]{ }
% \begin{frame}\transwipe
%  \frametitle{Sujet 2 : IHM (interface homme-machine) virtuelle}
%  \framesubtitle{Introduction}
%   \begin{itemize}[<+->]
%     \item Difficulté : de 3 à 5 étoiles
%       \begin{itemize}[<+->]
% \item Algorithmes avancés d'I.A ( réseaux de neurones, réseau bayésien, chaîne de Markov cachées, ... )
% \item Pas toujours au point
% \item Pas toujours libre
% \item Travail de recherche ( lire articles , ... )
%       \end{itemize}
%     \item Durée : de 2 à 5 étoiles
%     \item Langages : ?
%   \end{itemize}
% \end{frame}
% 
% \begin{frame}\transwipe
%  \frametitle{Sujet 2 : IHM (interface homme-machine) virtuelle}
%  \framesubtitle{C'est quoi?}
%  \begin{center}
%  \includegraphics[height=196px]{data/holo.jpg}
%   \end{center}
% \end{frame}
% 
% \begin{frame}\transwipe
%  \frametitle{Sujet 2 : IHM (interface homme-machine) virtuelle}
%  \framesubtitle{Structure}
%  \begin{center}
%  \includegraphics[height=196px]{data/S2_schema.jpg}
%   \end{center}
% \end{frame}
% 
% \begin{frame}\transwipe
%  \frametitle{Sujet 2 : IHM (interface homme-machine) virtuelle}
%  \framesubtitle{Intérêts}
%  \begin{center}
%  \includegraphics[height=130px]{data/S2_schema.jpg}
%   \end{center}
%     \begin{itemize}
%       \item<2-> intéractions avec ordinateur ( musique, recherche, ...)
%       \item<3-> intéractions avec monde réel ( contrôle maison, ...)
%     \end{itemize}
% \end{frame}
% 
% \begin{frame}\transwipe
%  \frametitle{Sujet 2 : IHM (interface homme-machine) virtuelle}
%  \framesubtitle{Conclusion}
%   \begin{itemize}[<+->]
%     \item Très enrichissant au niveau de l'IA
%     \item Bonne introduction à la recherche
%     \item Peu de possibilitées de financement ( publicité, vente logiciel, ... )
%     \item Long, compliqué
%     \item \bf{Incomplet / Imparfait}
%   \end{itemize}
%   
%    \uncover<6->{
%     \begin{alertblock}{ Chances de réussite }
% 20\%
%     \end{alertblock}
%    }
%    
% \end{frame}
% 
% 
% \section[Sujet 3 : Serveur]{Sujet 3 : Serveur}
% 
% \subsection[ ]{ }
% \begin{frame}\transwipe
%  \frametitle{Sujet 3 : Vulgarisation de serveur (+décentralisation)}
%  \framesubtitle{Introduction}
%   \begin{itemize}[<+->]
%     \item Difficulté : 2 étoiles
%       \begin{itemize}[<+->]
% \item Savoir configurer les différents outils
% \item Transactions sécurisées
%       \end{itemize}
%     \item Durée : de 2 à 4 étoiles
%     \item Langages : JSE/JEE/C
%       \begin{itemize}[<+->]
% \item Bash
%       \end{itemize}
%   \end{itemize}
% \end{frame}
% 
% \begin{frame}\transwipe
%  \frametitle{Sujet 3 : Vulgarisation de serveur (+décentralisation)}
%  \framesubtitle{Structure}
%  \begin{center}
%   \includegraphics[height=196px]{data/S3_schema.jpg}
%  \end{center}
% \end{frame}
% 
% \begin{frame}\transwipe
%  \frametitle{Sujet 3 : Vulgarisation de serveur (+décentralisation)}
%  \framesubtitle{Aller plus loin}
%  Décentraliser les services
%  \begin{itemize}
%   \item<2-> Chat
%   \item<3-> Streaming
%   \item<4-> ...
%  \end{itemize}
% 
% \end{frame}
% 
% 
% \begin{frame}\transwipe
%  \frametitle{Sujet 3 : Vulgarisation de serveur (+décentralisation)}
%  \framesubtitle{Conclusion}
%   \begin{itemize}[<+->]
%     \item Peu enrichissant au niveau de l'algorithmique
%     \item Enrichissant sur les différents outils
%     \item Pas de possibilitées de financement
%     \item Peut être soutenu par une communauté libre
%     \item \bf{Motivation}
%   \end{itemize}
%   
%    \uncover<6->{
%     \begin{exampleblock}{ Chances de réussite }
% 75\%
%     \end{exampleblock}
%    }
%    
% \end{frame}
% 
% \section[Sujet 4 : Dialogues]{Sujet 4 : Dialogues}
% 
% \subsection[ ]{ }
% \begin{frame}\transwipe
%  \frametitle{Sujet 4 : Dialogues non instantanés par prévisions}
%  \framesubtitle{Introduction}
%   \begin{itemize}[<+->]
%     \item Difficulté : de 1 étoiles
%       \begin{itemize}[<+->]
% \item Mélange dot et javascript
%       \end{itemize}
%     \item Durée : de 1 étoile
%     \item Langages :
%       \begin{itemize}
%      \item Java ( JSF / JPA / ... )
%      \item Javascript ( Jquery / ... )
%       \end{itemize}
%   \end{itemize}
% \end{frame}
% 
% 
% \begin{frame}\transwipe
%  \frametitle{Sujet 4 : Dialogues non instantanés par prévisions}
%  \framesubtitle{Structure}
%  \begin{center}
%   \includegraphics[height=190px]{data/S4_schema.jpg}
%  \end{center}
% \end{frame}
% 
% 
% \begin{frame}\transwipe
%  \frametitle{Sujet 4 : Dialogues non instantanés par prévisions}
%  \framesubtitle{Conclusion}
%   \begin{itemize}[<+->]
%     \item Très peu enrichissant
%     \item Possibilitées de financement ( publicité, services, ... )
%     \item Financements faibles
%     \item \bf{Facilité}
%   \end{itemize}
%   
%    \uncover<5->{
%     \begin{exampleblock}{ Chances de réussite }
% 90\%
%     \end{exampleblock}
%    }
%    
% \end{frame}
% 
% 
% \section[Récapitulatif]{Récapitulatif}
% 
% \subsection[ ]{ }
% \begin{frame}\transwipe
%  \frametitle{Différences}
%  \begin{center}
%   \begin{tabular}{|l|c|c|c|}
%   \hline
%   Sujets & Intéressant & Financements & Réussite \\
%   \hline
%   Sujet 1 : Jeu vidéo & 3.5 & 4 & 60\%
%   \\
%   \hline
%   Sujet 2 : IHM & 4 & 1 & 20\%
%   \\
%   \hline
%   Sujet 3 : Serveur & 2 & 0 & 75\%
%   \\
%   \hline
%   Sujet 4 : Dialogues & 1 & 3 & 90\%
%   \\
%   \hline
%   \end{tabular}
%  \end{center}
% \end{frame}

\end{document}


