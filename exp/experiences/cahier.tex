\documentclass[a4paper,12pt]{article}
\usepackage[utf8x]{inputenc} %commentaire
\usepackage[francais]{babel} %FR
\usepackage[T1]{fontenc}
\usepackage{fancyhdr}

\usepackage[pdftex]{graphicx} % img
\usepackage{wrapfig}
\usepackage{float}

\usepackage{algpseudocode}

\usepackage[top=2cm, bottom=2cm, left=2.5cm, right=2.5cm]{geometry} %Réduire les marges


% Style Page
\pagestyle{fancy} % entêtes
% \fancyhf{}

\setlength{\headheight}{15pt}
\renewcommand{\sectionmark}[1]{\markright{#1}} 
\lhead{ \rightmark }

\sloppy % ne pas faire déborder les lignes dans la marge


\title{
  \textbf{Cahier d'expériences}
  \\[5cm]
  Exploration de la notion de méta-apprentissage
  \\[3cm]
  \textit{
  Dans quelle mesure un système apprenant peut prendre conscience de ses performances
  et altérer son comportement ?}
}


\author{
  \\[3cm]
  Yann Boniface, Alain Dutech, Nicolas Rougier \\
  Matthieu Zimmer}

\begin{document}
  \maketitle
  \newpage

  \begin{center}{\Large Plan }\end{center}
  \tableofcontents
  \newpage

  \section{Expérience A1} \label{expA1}
  \subsection{Objectif}
    Comprendre de quelles manières peuvent émerger des représentations et méta-représentations dans 
    un réseau de neurone connexionniste, plus particulièrement sur des perceptrons multicouches.
    
    
    Reproduction et approfondissement des résultats de la première expérience de l'article 
    \cite{Cleeremans_2007}. 

  \subsection{Architecture}
    \paragraph{Description}
      Un premier réseau de perceptron multicouche apprend à discrétiser des chiffres représentés
      par 20 neurones d'entrées. Il est composé d'une couche cachée de 5 neurones.
      
      Un second réseau de perceptron multicouche apprend à dupliquer toutes les couches du premier
      réseau en n'ayant que sa couche cachée en entrée.
      
      L'apprentissage du second réseau, n'affecte pas les poids entre la couche d'entrée et la 
      couche cachée du premier réseau.

    \paragraph{Schéma}
      \begin{center}
	\includegraphics[width=220px]{data/expA1/schema.png}
      \end{center}
      
    \paragraph{Paramètres}
      \begin{center}
	\begin{tabular}{lr}
	  \begin{minipage}{220px}
	    \begin{itemize}
	      \item momentum : 0.9 sur les 2 réseau
	      \item taux d'apprentissage : 0.1 sur les 2 réseau
	      \item 10 chiffres différents présentés
	      \item apprentissage 10 (formes) x 1000 (époques)
	      \item utilisation de biais
	    \end{itemize}
	  \end{minipage}
	  &
	  \begin{minipage}{205px}
	    \begin{itemize}
	      \item poids initialisés sur [-0.25 ; 0.25]
	      \item taux d'apprentissage constant
	      \item entrées valent 0 ou 1
	      \item sigmoïde à température 1
	    \end{itemize}
	  \end{minipage}
	\end{tabular}
      \end{center}

  
  \newpage
  \subsection{Résultats}
    \paragraph{Principaux}
      Analyse des performances
      \begin{center}
	\begin{tabular}{lr}
	  \hspace*{-1cm}
	  \includegraphics[width=250px]{data/expA1/rms.png}
	  &
	  \includegraphics[width=250px]{data/expA1/err.png} 
	\end{tabular}
      \end{center}
      \subparagraph{Notes}
	\begin{itemize}
	  \item formule utilisée pour RMS (cf. Formules~\nameref{rms})
	  \item les courbes SoN layer représentent les erreurs (du second réseaux) sur les couches à reproduire 
	  \item la courbe RMS verte (SoN) est la somme des 3 courbes SoN layer
	\end{itemize}
      \subparagraph{Conclusion}
	\begin{itemize}
	  \item le premier réseau réussit à apprendre sa tâche de classification
	  \item la couche cachée et la couche de sortie ne posent aucun problèmes d'apprentissage
	  \item les performances du second réseau dépendent principalement de sa capacité à reproduire les entrées
	  \item le second réseau apprend plus rapidement que le premier
	\end{itemize}
    \paragraph{Secondaires}
      Discrétisation de la couche cachée du premier réseau
      \begin{center}
	\begin{tabular}{lr}
	  \hspace*{-1cm}
	  \includegraphics[width=250px]{data/expA1/discretize_cloud.png}
	  &
	  \includegraphics[width=250px]{data/expA1/discretize.png} 
	\end{tabular}
      \end{center} 
      \subparagraph{Notes}
	\begin{itemize}
	  \item une couleur équivaut à un chiffre présenté
	  \item une valeur discretisée correspond à un certain encodage de la couche cachée (cf Algorithmes~\nameref{discretize})
	\end{itemize}
      \subparagraph{Conclusion}
	Les neurones se stabilisent très rapidement (autour de la 50\up{ième} époque en moyenne), 
	le tout permettant au second réseau d'avoir des entrées très peu variables, favorisant
	son apprentissage.
    \paragraph{Secondaires}
      Représentations au travers des poids du premier réseau
      \begin{center}
	Couche cachée \\
	\includegraphics[width=250px]{data/expA1/representation_hidden.png}
      \end{center}
      \begin{center}
	Couche de sortie \\
	\includegraphics[width=250px]{data/expA1/representation.png}
      \end{center} 
      \subparagraph{Notes}
	\begin{itemize}
	  \item plus une case est noire, plus sa présence est importante pour le chiffre en question
	  \item plus une case est blanche, plus son absence est importante
	\end{itemize}
      \subparagraph{Conclusion}
	Il est assez difficile d'y distinquer les chiffres, mais cela semble suffisant pour le réseau
	qui a un taux de reconnaissance de 100\%.
    \paragraph{Secondaires}
      Prototypes à l'intérieur de la première partie de la couche de sortie du second réseau
      \begin{center}
	\includegraphics[width=250px]{data/expA1/prototype.png}
      \end{center} 
      \subparagraph{Notes}
	\begin{itemize}
	  \item Il sagit de la moyenne des réponses du second réseaux sur toutes les entrées
	\end{itemize}
      \subparagraph{Conclusion}
	Le peu d'entrées permet l'apprentissage par-coeur de chaque forme.
	


  \subsection{Conclusion}
  
  

  \newpage 
  \subsection{Formules}
    \paragraph{RMS} \label{rms}
  Pour une époque $e$ :
  \begin{center}
    \begin{large}
    $ rms_{e} = \sqrt{ \frac{1}{n} \sum \limits_{i=1}^{n} 
    ( o_{i,e} - d_{i} )^2 } $
    \end{large}
  $ with \left\lbrace \begin{array}{lll} n : number\ of\ neurons\ on\ the\ output\ 
  layer\\o_{i,e} : value\ obtained\ for\ the\ i^{th}\ neuron\ at\ the\ e^{th}\ epoch\\d_{i} : 
  value\ desired \ for\ the\ i^{th}\ neuron\end{array} \right.$
  \end{center}
    \paragraph{Discrétisation} Pour la couche cachée $hiddenNeuron$ de $n$ neurones, un neurone
      pouvant être encodé par $number\_cutting$ valeurs différentes :
      \begin{center}
	$\sum \limits_{i=0}^{n} number\_cutting^{i} \times cutting(hiddenNeuron[i]) $
      \end{center}
      \subparagraph{Exemple}
	$400 \gets [0 ; 0,25 ]\ [0 ; 0,25 ]\  [0,25 ; 0,5 ]\  [0,5 ; 0,75 ]\  [0,25 ; 0,5 ]$ \\
	\hspace*{2.70cm}
	$400 \gets 0\times4^0 +   0\times4^1  +   1\times4^2   +  2\times4^3   +   1 \times4^4$
    \paragraph{Descente de gradient} \cite{Touzet_1992} \\
  Construction de l'erreur : 
    \begin{center}
      $y_{i} = f'(a_i) \times ( d_i - x_i ) \ si\ i\ neurone\ de\ sortie $ \\
      $y_{i} = f'(a_i) \times \sum \limits_{k} ( w_{ki} \times y_k )\ si\ i\ neurone\ cache $
    \end{center}
  Mise à jour des poids :
    \begin{center}
      $w_{ij}(t+1) = w_{ij}(t) + learning\_rate \times y_{i} \times x_j + momentum \times 
      (w_{ij}(t) - w_{ij}(t-1) )$
    \end{center}
  Variables : 
    \begin{center}
      $\left\lbrace \begin{array}{lll} 
	f : fonction\ sigmoide \\
	x_i : valeur\ du\ neurone\ i\\
	d_i : valeur\ desire pour\ le\ neurone\ i\\
	a_i : somme\ pondere\ des\ poids\ du\ neurone\ i
      \end{array} \right.$
    \end{center}
    
\bibliographystyle{../pre-rapport/apalike}
\bibliography{../pre-rapport/biblio}

\end{document}



