\section{Expérience C3} 
  \subsection{Objectif}
    Comprendre de quelles manières un réseau de neurone connexionniste peut parier sur ses propres résultats
    à partir de ses représentations personnelles.

    
    Réalisation de la seconde expérience de l'article \cite{Cleeremans_2007} sur des données réelles
    non linéairement séparables.
  
  
  \subsection{Architecture}
    \paragraph{Description}
      Un premier réseau de perceptron multicouche apprend à discrétiser des fleurs caractérisées
      par 4 neurones d'entrées qui représentent taille et largeur de la pétale et la sépale. 
      Il est composé d'une couche cachée de 5 ou 100 neurones.
      
      Un second réseau de perceptron multicouche apprend à parier sur la qualité de la réponse
      du premier réseau à partir de sa couche cachée.
      
      L'apprentissage du second réseau, n'affecte pas les poids entre la couche d'entrée et la 
      couche cachée du premier réseau.

    \paragraph{Schéma}
      \begin{center}
	\includegraphics[width=230px]{data/expC3/schema.png}
      \end{center}
      
    \paragraph{Paramètres}
      \begin{center}
	\begin{tabular}{lr}
	  \begin{minipage}{230px}
	    \begin{itemize}
	      \item momentum : 0.5 sur le premier réseau
	      \item momentum : 0. sur le second réseau
	      \item taux d'apprentissage : 0.15 sur les 2 réseaux
	      \item \textbf{150 formes} de fleurs différents présentées (shuffle) \cite{Iris}
	      \item apprentissage 50 (formes) x 300 (époques)
	      \item utilisation de biais
	      
	    \end{itemize}
	  \end{minipage}
	  &
	  \begin{minipage}{230px}
	    \begin{itemize}
	      \item poids initialisés sur [-1 ; 1] pour le premier réseau
	      \item poids initialisés sur [-0.25 ; 0.25] pour le second réseau
	      \item taux d'apprentissage constant
	      \item entrées réelles sur [0 ; 1]
	      \item sigmoïde à température 1
	    \end{itemize}
	  \end{minipage}
	\end{tabular}
      \end{center}

  
  \newpage
  \subsection{Résultats}
    \paragraph{Principaux}
      Analyse des performances
      \begin{center}
	\begin{tabular}{lr}
	  \hspace*{-1cm}
	  \includegraphics[width=250px]{data/expC3/perf_5.png}
	  &
	  \includegraphics[width=250px]{data/expC3/perf_100.png} 
	\end{tabular}
      \end{center}
      \subparagraph{Notes}
	\begin{itemize}
	  \item la courbe rouge représentent le taux de paris hauts du second réseau
	\end{itemize}
      \subparagraph{Conclusion}
	\begin{itemize}
	  \item dans les 2 cas, le premier réseau réussit à apprendre sa tâche de classification
	  \item dans les 2 cas, le second réseau n'arrive pas à tirer parti de ses représentations, 
	  il se contente simplement de parier haut au bout d'un moment
	\end{itemize}
    \paragraph{Secondaires}
      RMS
      \begin{center}
	\begin{tabular}{lr}
	  \hspace*{-1cm}
	  \includegraphics[width=250px]{data/expC3/rms_5.png}
	  &
	  \includegraphics[width=250px]{data/expC3/rms_100.png} 
	\end{tabular}
      \end{center} 
      \subparagraph{Notes}
	\begin{itemize}
	  \item formule utilisée pour RMS (cf. Formules~\nameref{rms})
	\end{itemize}
      \subparagraph{Conclusion}
	Il n'y a plus de pique du second réseau comme dans les \nameref{expC1} et \nameref{expC2}.


  \subsection{Conclusion}
    Lors du passage sur des données réelles non linéairement séparables, l'utilité du second réseau
    s'écroule, les représentations ne sont plus exploitées.
    
    Pour tenter de résoudre ce problème, l'\nameref{expC4} augmente le nombre de couche du premier réseau.

  \newpage 
  \subsection{Formules}
    \paragraph{RMS} \label{rms}
  Pour une époque $e$ :
  \begin{center}
    \begin{large}
    $ rms_{e} = \sqrt{ \frac{1}{n} \sum \limits_{i=1}^{n} 
    ( o_{i,e} - d_{i} )^2 } $
    \end{large}
  $ with \left\lbrace \begin{array}{lll} n : number\ of\ neurons\ on\ the\ output\ 
  layer\\o_{i,e} : value\ obtained\ for\ the\ i^{th}\ neuron\ at\ the\ e^{th}\ epoch\\d_{i} : 
  value\ desired \ for\ the\ i^{th}\ neuron\end{array} \right.$
  \end{center}
    \paragraph{Descente de gradient} \cite{Touzet_1992} \\
  Construction de l'erreur : 
    \begin{center}
      $y_{i} = f'(a_i) \times ( d_i - x_i ) \ si\ i\ neurone\ de\ sortie $ \\
      $y_{i} = f'(a_i) \times \sum \limits_{k} ( w_{ki} \times y_k )\ si\ i\ neurone\ cache $
    \end{center}
  Mise à jour des poids :
    \begin{center}
      $w_{ij}(t+1) = w_{ij}(t) + learning\_rate \times y_{i} \times x_j + momentum \times 
      (w_{ij}(t) - w_{ij}(t-1) )$
    \end{center}
  Variables : 
    \begin{center}
      $\left\lbrace \begin{array}{lll} 
	f : fonction\ sigmoide \\
	x_i : valeur\ du\ neurone\ i\\
	d_i : valeur\ desire pour\ le\ neurone\ i\\
	a_i : somme\ pondere\ des\ poids\ du\ neurone\ i
      \end{array} \right.$
    \end{center}
    
\bibliographystyle{../pre-rapport/apalike}
\bibliography{../pre-rapport/biblio}
