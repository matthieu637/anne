\documentclass[a4paper,11pt]{article}
\usepackage[utf8x]{inputenc} %commentaire

\usepackage[francais]{babel} %FR
\usepackage[T1]{fontenc}
\usepackage{listings}
\usepackage{color}
\usepackage{verbatim}
\usepackage[top=2cm, bottom=2cm, left=2cm, right=2cm]{geometry} %Réduire les marges



\lstset{%
 	float=hbp,basicstyle=\footnotesize\ttfamily\color{white},%
 	columns=fixed,tabsize=4,frame=single,extendedchars=false,%
	showspaces=false,showstringspaces=false,numbers=none,%
	numberstyle=\tiny\ttfamily\color{black},
 	breaklines=true,breakindent=3em,breakautoindent=true,%
 	captionpos=t,xleftmargin=-1em,xrightmargin=-1em,lineskip=0pt,%
 	numbersep=1em,backgroundcolor=\color{black},%
 	keywordstyle=\bfseries\color{yellow}
}


\title{Comment utiliser le simulateur}
\author{}
\date{}


\begin{document}
\maketitle

\section{En tant que commande}

\subsection{Pour discrétiser un espace d'entrée à 2 dimensions}

\begin{lstlisting}[language={sh}]
java -jar mlp-simu.jar -x
\end{lstlisting}

Dans l'onglet \textbf{Params}, il suffit de renseigner les différents paramètres et valider.
La ligne ''affichage du réseau'' correspond au nombre de colonnes pour une couche du réseau 
(sachant que la couche d'entrée n'est pas comptée).

Dans l'onglet \textbf{Réseau}, la structure du réseau est visible et il faut maintenant choisir
à quelles entrées on assigne une croix ou un rond en cliquant dessus, puis lancer l'apprentissage.

Avant de regarder les différents neurones il faut vérifier que l'apprentissage a réussi 
dans l'onglet \textbf{Courbes}. Si ce n'est pas le cas, il faudra jouer sur les paramètres.\\
Note : Le nombre d'époques représente le nombre de fois qu'on présente une entrée au réseau.

\subsection{Pour simuler \& afficher un perceptron multicouche}

\begin{lstlisting}[language={sh}]
java -jar mlp-simu.jar <liste taille couche> <liste taille affichage couche> <fonction transfert> <fichier corpus> <taux apprentissage> <epoques> <precision courbe>
\end{lstlisting}

Par exemple,
\begin{lstlisting}[language={sh}]
java -jar mlp-simu.jar 256,50,10 16,5,1 sigmoid01 digits.corpus 0.15 55000 250
\end{lstlisting}

Le fichier corpus doit simplement être structuré par un couple d'entrée/sortie par ligne.

Un tier des données du corpus seront considérées comme base de test, le reste en tant
qu'apprentissage.

\section{En tant que bibliothèque}

Lier le fichier jar au projet.

\subsection{Pour simuler \& afficher un perceptron multicouche}

Exemple de simulation de la fonction XOR. Avec une couche intermédiaire contenant 2 neurones, 
sur 1000 époques, avec une fonction sigmoid sur [0 ; 1].

\begin{lstlisting}
public static void AND() {
	Simulation s = new Simulation(Arrays.asList(2, 2, 1), new Sigmoid01());
	s.ajouterDonneeApp(Arrays.asList(0., 0., 0.));
	s.ajouterDonneeApp(Arrays.asList(1., 0., 1.));
	s.ajouterDonneeApp(Arrays.asList(0., 1., 1.));
	s.ajouterDonneeApp(Arrays.asList(1., 1., 0.));
	s.lancer(0.15F, 1000 * 4, 1);
	new SimulationUI(s, Arrays.asList(1, 1, 1)).afficher();
}
\end{lstlisting}

\newpage
Exemple de simulation avec un corpus externe.
\begin{lstlisting}
	public static void digits() throws IOException {
		Simulation s = new Simulation(Arrays.asList(16 * 16, 50, 10), new Sigmoid01());
		s.corpusDepuis("digits.corpus");
		s.lancer(0.15F, 55000, 250);
		new SimulationUI(s, Arrays.asList(16, 5, 1)).afficher();
	}
\end{lstlisting}
	
\end{document}
